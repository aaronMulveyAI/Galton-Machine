% Options for packages loaded elsewhere
\PassOptionsToPackage{unicode}{hyperref}
\PassOptionsToPackage{hyphens}{url}
%
\documentclass[
]{article}
\title{Assigment 3}
\author{Aaron Mulvey}
\date{10/7/2022}

\usepackage{amsmath,amssymb}
\usepackage{lmodern}
\usepackage{iftex}
\ifPDFTeX
  \usepackage[T1]{fontenc}
  \usepackage[utf8]{inputenc}
  \usepackage{textcomp} % provide euro and other symbols
\else % if luatex or xetex
  \usepackage{unicode-math}
  \defaultfontfeatures{Scale=MatchLowercase}
  \defaultfontfeatures[\rmfamily]{Ligatures=TeX,Scale=1}
\fi
% Use upquote if available, for straight quotes in verbatim environments
\IfFileExists{upquote.sty}{\usepackage{upquote}}{}
\IfFileExists{microtype.sty}{% use microtype if available
  \usepackage[]{microtype}
  \UseMicrotypeSet[protrusion]{basicmath} % disable protrusion for tt fonts
}{}
\makeatletter
\@ifundefined{KOMAClassName}{% if non-KOMA class
  \IfFileExists{parskip.sty}{%
    \usepackage{parskip}
  }{% else
    \setlength{\parindent}{0pt}
    \setlength{\parskip}{6pt plus 2pt minus 1pt}}
}{% if KOMA class
  \KOMAoptions{parskip=half}}
\makeatother
\usepackage{xcolor}
\IfFileExists{xurl.sty}{\usepackage{xurl}}{} % add URL line breaks if available
\IfFileExists{bookmark.sty}{\usepackage{bookmark}}{\usepackage{hyperref}}
\hypersetup{
  pdftitle={Assigment 3},
  pdfauthor={Aaron Mulvey},
  hidelinks,
  pdfcreator={LaTeX via pandoc}}
\urlstyle{same} % disable monospaced font for URLs
\usepackage[margin=1in]{geometry}
\usepackage{color}
\usepackage{fancyvrb}
\newcommand{\VerbBar}{|}
\newcommand{\VERB}{\Verb[commandchars=\\\{\}]}
\DefineVerbatimEnvironment{Highlighting}{Verbatim}{commandchars=\\\{\}}
% Add ',fontsize=\small' for more characters per line
\usepackage{framed}
\definecolor{shadecolor}{RGB}{248,248,248}
\newenvironment{Shaded}{\begin{snugshade}}{\end{snugshade}}
\newcommand{\AlertTok}[1]{\textcolor[rgb]{0.94,0.16,0.16}{#1}}
\newcommand{\AnnotationTok}[1]{\textcolor[rgb]{0.56,0.35,0.01}{\textbf{\textit{#1}}}}
\newcommand{\AttributeTok}[1]{\textcolor[rgb]{0.77,0.63,0.00}{#1}}
\newcommand{\BaseNTok}[1]{\textcolor[rgb]{0.00,0.00,0.81}{#1}}
\newcommand{\BuiltInTok}[1]{#1}
\newcommand{\CharTok}[1]{\textcolor[rgb]{0.31,0.60,0.02}{#1}}
\newcommand{\CommentTok}[1]{\textcolor[rgb]{0.56,0.35,0.01}{\textit{#1}}}
\newcommand{\CommentVarTok}[1]{\textcolor[rgb]{0.56,0.35,0.01}{\textbf{\textit{#1}}}}
\newcommand{\ConstantTok}[1]{\textcolor[rgb]{0.00,0.00,0.00}{#1}}
\newcommand{\ControlFlowTok}[1]{\textcolor[rgb]{0.13,0.29,0.53}{\textbf{#1}}}
\newcommand{\DataTypeTok}[1]{\textcolor[rgb]{0.13,0.29,0.53}{#1}}
\newcommand{\DecValTok}[1]{\textcolor[rgb]{0.00,0.00,0.81}{#1}}
\newcommand{\DocumentationTok}[1]{\textcolor[rgb]{0.56,0.35,0.01}{\textbf{\textit{#1}}}}
\newcommand{\ErrorTok}[1]{\textcolor[rgb]{0.64,0.00,0.00}{\textbf{#1}}}
\newcommand{\ExtensionTok}[1]{#1}
\newcommand{\FloatTok}[1]{\textcolor[rgb]{0.00,0.00,0.81}{#1}}
\newcommand{\FunctionTok}[1]{\textcolor[rgb]{0.00,0.00,0.00}{#1}}
\newcommand{\ImportTok}[1]{#1}
\newcommand{\InformationTok}[1]{\textcolor[rgb]{0.56,0.35,0.01}{\textbf{\textit{#1}}}}
\newcommand{\KeywordTok}[1]{\textcolor[rgb]{0.13,0.29,0.53}{\textbf{#1}}}
\newcommand{\NormalTok}[1]{#1}
\newcommand{\OperatorTok}[1]{\textcolor[rgb]{0.81,0.36,0.00}{\textbf{#1}}}
\newcommand{\OtherTok}[1]{\textcolor[rgb]{0.56,0.35,0.01}{#1}}
\newcommand{\PreprocessorTok}[1]{\textcolor[rgb]{0.56,0.35,0.01}{\textit{#1}}}
\newcommand{\RegionMarkerTok}[1]{#1}
\newcommand{\SpecialCharTok}[1]{\textcolor[rgb]{0.00,0.00,0.00}{#1}}
\newcommand{\SpecialStringTok}[1]{\textcolor[rgb]{0.31,0.60,0.02}{#1}}
\newcommand{\StringTok}[1]{\textcolor[rgb]{0.31,0.60,0.02}{#1}}
\newcommand{\VariableTok}[1]{\textcolor[rgb]{0.00,0.00,0.00}{#1}}
\newcommand{\VerbatimStringTok}[1]{\textcolor[rgb]{0.31,0.60,0.02}{#1}}
\newcommand{\WarningTok}[1]{\textcolor[rgb]{0.56,0.35,0.01}{\textbf{\textit{#1}}}}
\usepackage{graphicx}
\makeatletter
\def\maxwidth{\ifdim\Gin@nat@width>\linewidth\linewidth\else\Gin@nat@width\fi}
\def\maxheight{\ifdim\Gin@nat@height>\textheight\textheight\else\Gin@nat@height\fi}
\makeatother
% Scale images if necessary, so that they will not overflow the page
% margins by default, and it is still possible to overwrite the defaults
% using explicit options in \includegraphics[width, height, ...]{}
\setkeys{Gin}{width=\maxwidth,height=\maxheight,keepaspectratio}
% Set default figure placement to htbp
\makeatletter
\def\fps@figure{htbp}
\makeatother
\setlength{\emergencystretch}{3em} % prevent overfull lines
\providecommand{\tightlist}{%
  \setlength{\itemsep}{0pt}\setlength{\parskip}{0pt}}
\setcounter{secnumdepth}{-\maxdimen} % remove section numbering
\ifLuaTeX
  \usepackage{selnolig}  % disable illegal ligatures
\fi

\begin{document}
\maketitle

\#\#Question 1 (Median)

In the last homework assignment, you calculated the expectation and the
variance of the Beta distribution with parameters a=6 and b=2. This time
around, find the median. Give a numerical solution without using pbeta
and check graphically if your solution makes sense. Tip: Use uniroot in
R.

I am going to plot the CFD of the beta distribution.

\begin{Shaded}
\begin{Highlighting}[]
\NormalTok{alpha }\OtherTok{=} \DecValTok{6}
\NormalTok{beta }\OtherTok{=} \DecValTok{2}

\NormalTok{secuencia }\OtherTok{=} \FunctionTok{seq}\NormalTok{(}\DecValTok{0}\NormalTok{, }\DecValTok{1}\NormalTok{, }\AttributeTok{by =} \FloatTok{0.001}\NormalTok{)}

\FunctionTok{plot}\NormalTok{(secuencia, }\FunctionTok{pbeta}\NormalTok{(secuencia, }\DecValTok{2}\NormalTok{,}\DecValTok{6}\NormalTok{), }\AttributeTok{xlab=}\StringTok{"X"}\NormalTok{,}
     \AttributeTok{ylab =} \StringTok{"Beta CDF"}\NormalTok{, }\AttributeTok{type =} \StringTok{"l"}\NormalTok{, }\AttributeTok{col =} \StringTok{"blue"}\NormalTok{)}
\end{Highlighting}
\end{Shaded}

\includegraphics{Script_files/figure-latex/unnamed-chunk-1-1.pdf} The
median of the beta distribution is a real number that F(X \textless{} m)
= 0.5 and F(X \textgreater{} m) = 0.5, so I will add some lines for see
where the median should be.

\includegraphics{Script_files/figure-latex/unnamed-chunk-2-1.pdf} Now I
want to get that point in the function, but my function is behind 0 so I
will subtract 0.5 so that I can get root.

\includegraphics{Script_files/figure-latex/unnamed-chunk-3-1.pdf} Next I
will use bisection method for getting the roots of my function.

\includegraphics{Script_files/figure-latex/unnamed-chunk-4-1.pdf} If I
continue like this I will reach the root therefore my solution.

But we cannot use pbeta so I will create my own method that will
integrate the beta distribution for the given x so that I am not using
pbeta

First I am going to create a function that integrates the beta function
for a given X.

\begin{Shaded}
\begin{Highlighting}[]
\NormalTok{BetaDistribution }\OtherTok{\textless{}{-}} \ControlFlowTok{function}\NormalTok{(x, alpha, beta)\{}
  \FunctionTok{return}\NormalTok{ (}\DecValTok{1} \SpecialCharTok{{-}} \FunctionTok{integrate}\NormalTok{(dbeta, }\AttributeTok{lower=}\DecValTok{0}\NormalTok{, }\AttributeTok{upper=}\NormalTok{x, alpha, beta)}\SpecialCharTok{$}\NormalTok{value)}
\NormalTok{\}}
\end{Highlighting}
\end{Shaded}

Now I am going to use a bisection method for geting the root

\begin{Shaded}
\begin{Highlighting}[]
\NormalTok{median }\OtherTok{\textless{}{-}} \ControlFlowTok{function}\NormalTok{(alpha, beta) \{}
  \FunctionTok{return}\NormalTok{(}\FunctionTok{uniroot}\NormalTok{(}\ControlFlowTok{function}\NormalTok{(x) }\FunctionTok{BetaDistribution}\NormalTok{(x,alpha,beta) }\SpecialCharTok{{-}} \FloatTok{0.5}\NormalTok{, }\AttributeTok{lower =} \DecValTok{0}\NormalTok{, }\AttributeTok{upper =} \DecValTok{1}\NormalTok{)}\SpecialCharTok{$}\NormalTok{root)}
\NormalTok{\}}

\FunctionTok{median}\NormalTok{(alpha, beta)}
\end{Highlighting}
\end{Shaded}

\begin{verbatim}
## [1] 0.7714982
\end{verbatim}

I saw an aproximation for the median that is:

\begin{Shaded}
\begin{Highlighting}[]
\NormalTok{aproximation }\OtherTok{\textless{}{-}} \ControlFlowTok{function}\NormalTok{(alpha,beta)\{}
    \FunctionTok{return}\NormalTok{ ((alpha }\SpecialCharTok{{-}} \DecValTok{1}\SpecialCharTok{/}\DecValTok{3}\NormalTok{) }\SpecialCharTok{/}\NormalTok{ (alpha }\SpecialCharTok{+}\NormalTok{ beta}\DecValTok{{-}2}\SpecialCharTok{/}\DecValTok{3}\NormalTok{))}
\NormalTok{\}}

\FunctionTok{aproximation}\NormalTok{(alpha, beta)}
\end{Highlighting}
\end{Shaded}

\begin{verbatim}
## [1] 0.7727273
\end{verbatim}

Now I will calculate the error of the aproximation

\begin{Shaded}
\begin{Highlighting}[]
\FunctionTok{sqrt}\NormalTok{((}\FunctionTok{aproximation}\NormalTok{(alpha, beta) }\SpecialCharTok{{-}} \FunctionTok{median}\NormalTok{(alpha, beta))}\SpecialCharTok{\^{}}\DecValTok{2}\NormalTok{) }
\end{Highlighting}
\end{Shaded}

\begin{verbatim}
## [1] 0.001229036
\end{verbatim}

\#\#Question 2

Answer the question in topic 8 on slide 14: Can you use the central
limit theorem to explain why the histogram produced should look
approximately Normal? Give a mathematical explanation and provide the
parameters of the Normal distribution.

Can you use the central limit theorem to explain why the histogram
produced should look approximately Normal?

When one ball reach the triangle, we can see that the ball has two
possibilities, either go to the left or go to the right. So the ball has
to take one direction (D), and the probability of taking one direction
or the other is a Bernoulli trial.

\[ 
D=\left\{ \begin{array}{ll} 0 & \text{if }x = \text{Left } \\  
1 & \text{if }x = \text{Rigth }
\end{array}
\right. ,  D ∼ Bernoulli(P = 0.5)
\]

\begin{Shaded}
\begin{Highlighting}[]
\NormalTok{t }\OtherTok{\textless{}{-}} \FunctionTok{seq}\NormalTok{(}\DecValTok{0}\NormalTok{, }\DecValTok{2} \SpecialCharTok{*} \FloatTok{3.1416}\NormalTok{, }\AttributeTok{length.out =} \DecValTok{100}\NormalTok{)}
\NormalTok{a }\OtherTok{\textless{}{-}} \DecValTok{0} 
\NormalTok{b }\OtherTok{\textless{}{-}} \DecValTok{1} 
\NormalTok{radio }\OtherTok{\textless{}{-}} \FloatTok{0.5}
\NormalTok{x }\OtherTok{\textless{}{-}}\NormalTok{ a }\SpecialCharTok{+} \FunctionTok{cos}\NormalTok{(t)}\SpecialCharTok{*}\NormalTok{radio}
\NormalTok{y }\OtherTok{\textless{}{-}}\NormalTok{ b }\SpecialCharTok{+} \FunctionTok{sin}\NormalTok{(t)}\SpecialCharTok{*}\NormalTok{radio}

\FunctionTok{plot}\NormalTok{(}\DecValTok{0}\NormalTok{, }\DecValTok{0}\NormalTok{, }\AttributeTok{asp =} \DecValTok{1}\NormalTok{, }\AttributeTok{type =} \StringTok{"n"}\NormalTok{, }\AttributeTok{xlim =} \FunctionTok{c}\NormalTok{(}\SpecialCharTok{{-}}\DecValTok{2}\NormalTok{, }\DecValTok{2}\NormalTok{), }\AttributeTok{ylim =} \FunctionTok{c}\NormalTok{(}\SpecialCharTok{{-}}\DecValTok{2}\NormalTok{, }\DecValTok{2}\NormalTok{), }\AttributeTok{xaxt =} \StringTok{"n"}\NormalTok{, }\AttributeTok{yaxt =} \StringTok{"n"}\NormalTok{)}
\FunctionTok{title}\NormalTok{(}\StringTok{"Bernoulli Trial"}\NormalTok{)}



\FunctionTok{segments}\NormalTok{(}\SpecialCharTok{{-}}\DecValTok{1}\NormalTok{, }\SpecialCharTok{{-}}\FloatTok{1.5}\NormalTok{, }\DecValTok{1}\NormalTok{,}\SpecialCharTok{{-}}\FloatTok{1.5}\NormalTok{)}
\FunctionTok{segments}\NormalTok{(}\SpecialCharTok{{-}}\DecValTok{1}\NormalTok{, }\SpecialCharTok{{-}}\FloatTok{1.5}\NormalTok{, }\DecValTok{0}\NormalTok{, }\DecValTok{0}\NormalTok{)}
\FunctionTok{segments}\NormalTok{(}\DecValTok{0}\NormalTok{, }\DecValTok{0}\NormalTok{, }\DecValTok{1}\NormalTok{,}\SpecialCharTok{{-}}\FloatTok{1.5}\NormalTok{)}
\FunctionTok{arrows}\NormalTok{(}\FloatTok{0.5}\NormalTok{, }\DecValTok{1}\NormalTok{, }\FloatTok{1.5}\NormalTok{, }\DecValTok{0}\NormalTok{, }\AttributeTok{lty =} \DecValTok{2}\NormalTok{)}
\FunctionTok{arrows}\NormalTok{(}\SpecialCharTok{{-}}\FloatTok{0.5}\NormalTok{, }\DecValTok{1}\NormalTok{, }\SpecialCharTok{{-}}\FloatTok{1.5}\NormalTok{, }\DecValTok{0}\NormalTok{, }\AttributeTok{lty =} \DecValTok{2}\NormalTok{)}
\FunctionTok{lines}\NormalTok{(x, y)}
\FunctionTok{text}\NormalTok{(}\AttributeTok{x =} \DecValTok{0}\NormalTok{, }\AttributeTok{y =} \DecValTok{1}\NormalTok{, }\AttributeTok{label=}\StringTok{\textquotesingle{}X\textquotesingle{}}\NormalTok{)}
\FunctionTok{text}\NormalTok{(}\SpecialCharTok{{-}}\DecValTok{2}\NormalTok{, }\SpecialCharTok{{-}}\FloatTok{0.25}\NormalTok{, }\AttributeTok{label=}\StringTok{\textquotesingle{}P(X = LEFT) = 0.5\textquotesingle{}}\NormalTok{)}
\FunctionTok{text}\NormalTok{(}\DecValTok{2}\NormalTok{, }\SpecialCharTok{{-}}\FloatTok{0.25}\NormalTok{, }\AttributeTok{label=}\StringTok{\textquotesingle{}P(X = RIGTH) = 0.5\textquotesingle{}}\NormalTok{)}
\end{Highlighting}
\end{Shaded}

\includegraphics{Script_files/figure-latex/unnamed-chunk-9-1.pdf} At the
end the ball will finish in one box or in the other one depending the
direction that takes.

Now lets assume that we have n levels of triangles, so the ball can
reach different boxes. The total number of boxes that the ball can reach
is n + 1.

\begin{Shaded}
\begin{Highlighting}[]
\NormalTok{t }\OtherTok{\textless{}{-}} \FunctionTok{seq}\NormalTok{(}\DecValTok{0}\NormalTok{, }\DecValTok{2} \SpecialCharTok{*} \FloatTok{3.1416}\NormalTok{, }\AttributeTok{length.out =} \DecValTok{100}\NormalTok{)}
\NormalTok{a }\OtherTok{\textless{}{-}} \DecValTok{2} 
\NormalTok{b }\OtherTok{\textless{}{-}} \FloatTok{0.6}
\NormalTok{radio }\OtherTok{\textless{}{-}} \FloatTok{0.5}
\NormalTok{x }\OtherTok{\textless{}{-}}\NormalTok{ a }\SpecialCharTok{+} \FunctionTok{cos}\NormalTok{(t)}\SpecialCharTok{*}\NormalTok{radio}
\NormalTok{y }\OtherTok{\textless{}{-}}\NormalTok{ b }\SpecialCharTok{+} \FunctionTok{sin}\NormalTok{(t)}\SpecialCharTok{*}\NormalTok{radio}

\FunctionTok{plot}\NormalTok{(}\DecValTok{0}\NormalTok{, }\DecValTok{0}\NormalTok{, }\AttributeTok{asp =} \DecValTok{1}\NormalTok{, }\AttributeTok{type =} \StringTok{"n"}\NormalTok{, }\AttributeTok{xlim =} \FunctionTok{c}\NormalTok{(}\SpecialCharTok{{-}}\DecValTok{2}\NormalTok{, }\DecValTok{2}\NormalTok{), }\AttributeTok{ylim =} \FunctionTok{c}\NormalTok{(}\SpecialCharTok{{-}}\DecValTok{2}\NormalTok{, }\DecValTok{2}\NormalTok{), }\AttributeTok{xaxt =} \StringTok{"n"}\NormalTok{, }\AttributeTok{yaxt =} \StringTok{"n"}\NormalTok{)}
\FunctionTok{title}\NormalTok{(}\StringTok{"n Bernoulli trials"}\NormalTok{)}
\FunctionTok{lines}\NormalTok{(x, y)}
\FunctionTok{text}\NormalTok{(}\AttributeTok{x =}\NormalTok{ a, }\AttributeTok{y =}\NormalTok{ b, }\AttributeTok{label=}\StringTok{\textquotesingle{}X\textquotesingle{}}\NormalTok{)}

\FunctionTok{arrows}\NormalTok{(}\FloatTok{2.5}\NormalTok{, }\FloatTok{0.6}\NormalTok{, }\FloatTok{3.75}\NormalTok{, }\SpecialCharTok{{-}}\DecValTok{1}\NormalTok{, }\AttributeTok{lty =} \DecValTok{2}\NormalTok{)}
\FunctionTok{arrows}\NormalTok{(}\FloatTok{1.5}\NormalTok{, }\FloatTok{0.6}\NormalTok{, }\FloatTok{0.25}\NormalTok{, }\SpecialCharTok{{-}}\DecValTok{1}\NormalTok{, }\AttributeTok{lty =} \DecValTok{2}\NormalTok{)}

\FunctionTok{segments}\NormalTok{(}\SpecialCharTok{{-}}\DecValTok{1}\NormalTok{, }\FloatTok{0.25}\NormalTok{, }\DecValTok{1}\NormalTok{,}\FloatTok{0.25}\NormalTok{)}
\FunctionTok{segments}\NormalTok{(}\SpecialCharTok{{-}}\DecValTok{1}\NormalTok{, }\FloatTok{0.25}\NormalTok{, }\DecValTok{0}\NormalTok{, }\FloatTok{1.75}\NormalTok{)}
\FunctionTok{segments}\NormalTok{(}\DecValTok{1}\NormalTok{, }\FloatTok{0.25}\NormalTok{, }\DecValTok{0}\NormalTok{, }\FloatTok{1.75}\NormalTok{)}

\FunctionTok{segments}\NormalTok{(}\SpecialCharTok{{-}}\DecValTok{3}\NormalTok{, }\SpecialCharTok{{-}}\FloatTok{1.5}\NormalTok{, }\SpecialCharTok{{-}}\DecValTok{1}\NormalTok{,}\SpecialCharTok{{-}}\FloatTok{1.5}\NormalTok{)}
\FunctionTok{segments}\NormalTok{(}\SpecialCharTok{{-}}\DecValTok{3}\NormalTok{, }\SpecialCharTok{{-}}\FloatTok{1.5}\NormalTok{, }\SpecialCharTok{{-}}\DecValTok{2}\NormalTok{, }\DecValTok{0}\NormalTok{)}
\FunctionTok{segments}\NormalTok{(}\SpecialCharTok{{-}}\DecValTok{1}\NormalTok{, }\SpecialCharTok{{-}}\FloatTok{1.5}\NormalTok{, }\SpecialCharTok{{-}}\DecValTok{2}\NormalTok{, }\DecValTok{0}\NormalTok{)}

\FunctionTok{segments}\NormalTok{(}\DecValTok{3}\NormalTok{, }\SpecialCharTok{{-}}\FloatTok{1.5}\NormalTok{, }\DecValTok{1}\NormalTok{,}\SpecialCharTok{{-}}\FloatTok{1.5}\NormalTok{)}
\FunctionTok{segments}\NormalTok{(}\DecValTok{3}\NormalTok{, }\SpecialCharTok{{-}}\FloatTok{1.5}\NormalTok{, }\DecValTok{2}\NormalTok{, }\DecValTok{0}\NormalTok{)}
\FunctionTok{segments}\NormalTok{(}\DecValTok{1}\NormalTok{, }\SpecialCharTok{{-}}\FloatTok{1.5}\NormalTok{, }\DecValTok{2}\NormalTok{, }\DecValTok{0}\NormalTok{)}
\end{Highlighting}
\end{Shaded}

\includegraphics{Script_files/figure-latex/unnamed-chunk-10-1.pdf}

As you can see there are 2 levels so n = 2 and the possible boxes where
the ball can go is n + 1 = 3.

The box in which that the ball will end is a sequence of n Bernoulli
trials. We can name the boxes like this:

\[ Box = \text{{0, 1,...,n}}\] So for know in which box is the ball what
we have to check is the sum of all the individual Bernoulli trials:

\[Ball_{Box} = \text{{0, 1, 1, 1, 0,..., n} } \] We can say that the box
in which the ball will end is a random variable X, that is distributed
as a binomial distribution. Where the n represents the set of boxes, p
the probability that is 0.5 and x select a box.

\[P(X = x) = \binom{n}{x} p^x(1-p)^{n-x}\] So the set of boxes are
distributed as:

\[X ∼ Binomial(n,p)\] \#\#Central limit theorem

Let \[ X_{1},X_{2},X_{3},...,X_{n} \] Be random independent variables
identically distributed with

\[E[X_{i}] = \mu  \text{    and    }  Var(X_{i}) = \sigma^2 < \infty\]
We say that,

\[ Z_{n} :=\frac{\sum X_i - n \mu}{\sigma \sqrt{n}} = \sqrt{n}\frac{\bar X - \mu}{\sigma}\]
Where, X is the sample mean. So when we make the limit of n going to
infinity the function Z converge to the normal distribution

\[\lim_{n \to \infty} P(Z_n < z) = \Phi(z) = \int_{-\infty}^{z} \frac{e^\frac{-x^2}{2}}{\sqrt{2\pi}}dx\]
When the n is big enough (n \textgreater{} 30) and p and q are far from
0 and 1 (p around 0.5) we can aproximate the binomial distribution to
the normal distribution.

\[X ∼ Binomial(n,p) \approx Normal(\mu, \sigma^2)\] Where mu is the
expectation:

\[\mu = \sum_{i=0}^{n} x_i P(X=x_i) = px_1+ px_2+px_3+...+px_n\] So we
take the p as comon factor:

\[ \mu = p(x_1+ x_2+x_3+...+x_n) =p \sum_{i=0}^{n} x_n = pn\]

So the mu = pn Now lets see the Var(x)

\[\sigma^2 = Var(X) = E(X^2) - E(X)^2\] \[E(X)^2 = n^2p^2\]
\[E(X^2) =  n^2p^2 +np(1-p)\]
\[\sigma^2 = Var(X) = E(X^2) - E(X)^2 =  n^2p^2 +np(1-p) - n^2p^2 = np(1-p)\]
so we can put in the X like:

\[X ∼ Normal(\mu, \sigma^2)\] \#\#Lets view the simulation of the Galton
Machine

\begin{Shaded}
\begin{Highlighting}[]
\FunctionTok{saveGIF}\NormalTok{(\{}
\NormalTok{  freq }\OtherTok{=} \FunctionTok{quincunx}\NormalTok{(}\AttributeTok{balls =} \DecValTok{200}\NormalTok{, }\AttributeTok{col.balls =} \FunctionTok{rainbow}\NormalTok{(}\DecValTok{200}\NormalTok{))}
  \FunctionTok{barplot}\NormalTok{(freq, }\AttributeTok{space =} \DecValTok{0}\NormalTok{, }\AttributeTok{main =} \StringTok{"Distribución Final de las Bolas"}\NormalTok{)}
\NormalTok{\}, }\AttributeTok{movie.name =} \StringTok{"quincunx.gif"}\NormalTok{, }\AttributeTok{ani.height =} \DecValTok{600}\NormalTok{, }\AttributeTok{ani.width =} \DecValTok{800}\NormalTok{)}
\end{Highlighting}
\end{Shaded}

\begin{verbatim}
## Output at: quincunx.gif
\end{verbatim}

\begin{verbatim}
## [1] TRUE
\end{verbatim}

\begin{Shaded}
\begin{Highlighting}[]
\NormalTok{knitr}\SpecialCharTok{::}\FunctionTok{include\_graphics}\NormalTok{(}\StringTok{"quincunx.gif"}\NormalTok{)}
\end{Highlighting}
\end{Shaded}

\includegraphics{quincunx.gif}

\hypertarget{lets-make-a-numeric-simulation}{%
\subsection{Lets make a numeric
simulation}\label{lets-make-a-numeric-simulation}}

\begin{Shaded}
\begin{Highlighting}[]
\NormalTok{Galton\_Simulation }\OtherTok{\textless{}{-}} \ControlFlowTok{function}\NormalTok{(B, n, p) \{}
  
\NormalTok{  Balls }\OtherTok{\textless{}{-}}\NormalTok{ B}
\NormalTok{  levels }\OtherTok{\textless{}{-}}\NormalTok{ n  }
\NormalTok{  direction }\OtherTok{\textless{}{-}} \FunctionTok{c}\NormalTok{(}\DecValTok{0}\NormalTok{,}\DecValTok{1}\NormalTok{)}
\NormalTok{  p }\OtherTok{\textless{}{-}}\NormalTok{ p  }
\NormalTok{  q }\OtherTok{\textless{}{-}} \DecValTok{1} \SpecialCharTok{{-}}\NormalTok{ p}
\NormalTok{  probability }\OtherTok{\textless{}{-}} \FunctionTok{c}\NormalTok{(p, q)}
\NormalTok{  Box\_n\_balls }\OtherTok{\textless{}{-}} \FunctionTok{c}\NormalTok{()}
  
  \ControlFlowTok{for}\NormalTok{ (i }\ControlFlowTok{in} \DecValTok{1}\SpecialCharTok{:}\NormalTok{Balls) \{}
\NormalTok{    Box\_1\_Ball }\OtherTok{\textless{}{-}} \FunctionTok{sum}\NormalTok{(}\FunctionTok{sample}\NormalTok{(direction, levels, probability, }\AttributeTok{replace =}\NormalTok{ T ))}
\NormalTok{    Box\_n\_balls[i] }\OtherTok{\textless{}{-}}\NormalTok{ Box\_1\_Ball }
\NormalTok{  \}}
  
\NormalTok{  a }\OtherTok{\textless{}{-}} \FloatTok{0.5}
  
\NormalTok{  Ylevel }\OtherTok{\textless{}{-}} \FunctionTok{paste}\NormalTok{(}\FunctionTok{toString}\NormalTok{(levels), }\StringTok{"Boxes,"}\NormalTok{, }\FunctionTok{toString}\NormalTok{(Balls), }\StringTok{"Balls"}\NormalTok{ ,}\AttributeTok{sep =} \StringTok{" "}\NormalTok{)}
  \FunctionTok{hist}\NormalTok{(Box\_n\_balls, }\AttributeTok{probability =}\NormalTok{ T, }\AttributeTok{breaks =} \FunctionTok{c}\NormalTok{(}\DecValTok{0} \SpecialCharTok{:}\NormalTok{ (levels }\SpecialCharTok{+} \FloatTok{0.5}\NormalTok{)), }
       \AttributeTok{col =} \StringTok{"blue"}\NormalTok{, }
       \AttributeTok{xlab =}\NormalTok{ Ylevel, }
       \AttributeTok{main =} \FunctionTok{paste}\NormalTok{(}\StringTok{"Histogram of"}\NormalTok{,Ylevel, }\AttributeTok{sep =} \StringTok{" "}\NormalTok{))}
\NormalTok{  secuencia }\OtherTok{=} \FunctionTok{seq}\NormalTok{(}\DecValTok{0}\NormalTok{, levels, }\AttributeTok{by =} \FloatTok{0.001}\NormalTok{)            }
\NormalTok{  normal }\OtherTok{\textless{}{-}} \FunctionTok{dnorm}\NormalTok{(secuencia, }\AttributeTok{mean =}\NormalTok{ levels }\SpecialCharTok{*}\NormalTok{ a, }\AttributeTok{sd =} \FunctionTok{sqrt}\NormalTok{(levels}\SpecialCharTok{*}\NormalTok{a}\SpecialCharTok{*}\NormalTok{(}\DecValTok{1}\SpecialCharTok{{-}}\NormalTok{a))) }\CommentTok{\#}
  \FunctionTok{lines}\NormalTok{(secuencia, normal, }\AttributeTok{col =} \StringTok{"red"}\NormalTok{)}
\NormalTok{\}}
\end{Highlighting}
\end{Shaded}

Now lets try to make the simulation for n = 5, n = 15 , n = 25 \& n = 35

\begin{Shaded}
\begin{Highlighting}[]
\NormalTok{B }\OtherTok{\textless{}{-}} \DecValTok{10000}
\NormalTok{p }\OtherTok{\textless{}{-}} \FloatTok{0.5}

\FunctionTok{par}\NormalTok{(}\AttributeTok{mfrow =} \FunctionTok{c}\NormalTok{(}\DecValTok{2}\NormalTok{, }\DecValTok{2}\NormalTok{))}

\FunctionTok{Galton\_Simulation}\NormalTok{(B , }\DecValTok{5}\NormalTok{, p)}

\FunctionTok{Galton\_Simulation}\NormalTok{(B , }\DecValTok{15}\NormalTok{, p)}

\FunctionTok{Galton\_Simulation}\NormalTok{(B , }\DecValTok{25}\NormalTok{, p)}

\FunctionTok{Galton\_Simulation}\NormalTok{(B , }\DecValTok{35}\NormalTok{, p)}
\end{Highlighting}
\end{Shaded}

\includegraphics{Script_files/figure-latex/unnamed-chunk-13-1.pdf}

\begin{Shaded}
\begin{Highlighting}[]
\FunctionTok{par}\NormalTok{(}\AttributeTok{mfrow =} \FunctionTok{c}\NormalTok{(}\DecValTok{1}\NormalTok{, }\DecValTok{1}\NormalTok{))}
\end{Highlighting}
\end{Shaded}

Last lets make same 4 simulations but each Bernoulli trial is not going
to be fair, I am going to put that the probability of the ball goes to
one side is of: P(X = Left) = 0.9

\begin{Shaded}
\begin{Highlighting}[]
\NormalTok{B }\OtherTok{\textless{}{-}} \DecValTok{10000}
\NormalTok{p }\OtherTok{\textless{}{-}} \FloatTok{0.9}

\FunctionTok{par}\NormalTok{(}\AttributeTok{mfrow =} \FunctionTok{c}\NormalTok{(}\DecValTok{2}\NormalTok{, }\DecValTok{2}\NormalTok{))}

\FunctionTok{Galton\_Simulation}\NormalTok{(B , }\DecValTok{5}\NormalTok{, p)}

\FunctionTok{Galton\_Simulation}\NormalTok{(B , }\DecValTok{15}\NormalTok{, p)}

\FunctionTok{Galton\_Simulation}\NormalTok{(B , }\DecValTok{25}\NormalTok{, p)}

\FunctionTok{Galton\_Simulation}\NormalTok{(B , }\DecValTok{35}\NormalTok{, p)}
\end{Highlighting}
\end{Shaded}

\includegraphics{Script_files/figure-latex/unnamed-chunk-14-1.pdf}

\begin{Shaded}
\begin{Highlighting}[]
\FunctionTok{par}\NormalTok{(}\AttributeTok{mfrow =} \FunctionTok{c}\NormalTok{(}\DecValTok{1}\NormalTok{, }\DecValTok{1}\NormalTok{))}
\end{Highlighting}
\end{Shaded}

\begin{Shaded}
\begin{Highlighting}[]
\NormalTok{Aarons\_Simulation }\OtherTok{\textless{}{-}} \ControlFlowTok{function}\NormalTok{(B, n) \{}
\NormalTok{  Balls }\OtherTok{\textless{}{-}}\NormalTok{ B}
\NormalTok{  levels }\OtherTok{\textless{}{-}}\NormalTok{ n}
  
  
\NormalTok{  Box\_n\_balls }\OtherTok{\textless{}{-}} \FunctionTok{numeric}\NormalTok{(Balls)}
  
  \ControlFlowTok{for}\NormalTok{ (i }\ControlFlowTok{in} \DecValTok{1}\SpecialCharTok{:}\NormalTok{Balls) \{}
  
\NormalTok{    Box\_1\_Ball }\OtherTok{\textless{}{-}} \FunctionTok{sum}\NormalTok{(}\FunctionTok{runif}\NormalTok{(levels) }\SpecialCharTok{\textless{}} \FloatTok{0.5}\NormalTok{)  }
\NormalTok{    Box\_n\_balls[i] }\OtherTok{\textless{}{-}}\NormalTok{ Box\_1\_Ball}
\NormalTok{  \}}
  
\NormalTok{  a }\OtherTok{\textless{}{-}} \FloatTok{0.5}
  
\NormalTok{  Ylevel }\OtherTok{\textless{}{-}} \FunctionTok{paste0}\NormalTok{(levels, }\StringTok{" Boxes, "}\NormalTok{, Balls, }\StringTok{" Balls"}\NormalTok{)}
  \FunctionTok{hist}\NormalTok{(Box\_n\_balls, }\AttributeTok{probability =} \ConstantTok{TRUE}\NormalTok{, }\AttributeTok{breaks =} \FunctionTok{seq}\NormalTok{(}\SpecialCharTok{{-}}\FloatTok{0.5}\NormalTok{, levels }\SpecialCharTok{+} \FloatTok{0.5}\NormalTok{, }\AttributeTok{by =} \DecValTok{1}\NormalTok{), }
       \AttributeTok{col =} \StringTok{"blue"}\NormalTok{, }
       \AttributeTok{xlab =}\NormalTok{ Ylevel, }
       \AttributeTok{main =} \FunctionTok{paste}\NormalTok{(}\StringTok{"Histogram of "}\NormalTok{, Ylevel))}
  

\NormalTok{  secuencia }\OtherTok{\textless{}{-}} \FunctionTok{seq}\NormalTok{(}\DecValTok{0}\NormalTok{, levels, }\AttributeTok{by =} \FloatTok{0.001}\NormalTok{)}
\NormalTok{  normal }\OtherTok{\textless{}{-}} \FunctionTok{dnorm}\NormalTok{(secuencia, }\AttributeTok{mean =}\NormalTok{ levels }\SpecialCharTok{*}\NormalTok{ a, }\AttributeTok{sd =} \FunctionTok{sqrt}\NormalTok{(levels }\SpecialCharTok{*}\NormalTok{ a }\SpecialCharTok{*}\NormalTok{ (}\DecValTok{1} \SpecialCharTok{{-}}\NormalTok{ a)))}
  \FunctionTok{lines}\NormalTok{(secuencia, normal, }\AttributeTok{col =} \StringTok{"red"}\NormalTok{)}
\NormalTok{\}}


\FunctionTok{Aarons\_Simulation}\NormalTok{(}\DecValTok{1000}\NormalTok{, }\DecValTok{35}\NormalTok{)}
\end{Highlighting}
\end{Shaded}

\includegraphics{Script_files/figure-latex/unnamed-chunk-15-1.pdf}

\end{document}
